\documentclass[a4paper,10pt]{article}

%% CONFIG

\IfFileExists{./.config.tex}{
\input{.config}
}
{
\def\isanonymous{0}
\def\isfullversion{1}
\def\buildexternal{1}
}

%% There is a package that you should include first: nag. Why first? Because it warns you about
%% obsolete commands and package. https://daniel-j-h.github.io/post/latex-a-modern-approach/
\usepackage[l2tabu,orthodox]{nag}

%% MISC PACKAGES

\usepackage{microtype}
\usepackage[bookmarksdepth=2,backref]{hyperref}
\usepackage{booktabs}  %% tables
\usepackage{comment}
\usepackage[inline]{enumitem}
\usepackage{nicematrix}

%% LINE NUMBERS

\usepackage{lineno}
\linenumbers{}

%% TODO NOTES

\usepackage[draft]{comments-bugs-todos}
\NewDocumentCommand\iwoo{s m o}{\IfBooleanTF{#1}{\cmnt*{iwoo}{#2}}{\cmnt{iwoo}{#2}[#3]}}

%% ANONYMOUS SUBMISSIONS

\usepackage{ifthen}
\newcommand{\anonymous}[2]{%
\ifthenelse{\equal{\isanonymous}{1}}%
{#1}%
{#2}%
}%

\newcommand{\fullversion}[2]{%
\ifthenelse{\equal{\isfullversion}{1}}%
{#1}%
{#2}%
}%

%% SYMBOLS & NOTATION

\usepackage{amsmath,amsfonts,amsthm,amssymb}
\usepackage{xspace}
\usepackage[lambda,landau,operators,probability,sets,adversary,logic,ff,primitives,complexity,oracles,asymptotics,keys]{cryptocode}

\renewcommand{\vec}[1]{\ensuremath{\mathbf{#1}}\xspace}
\newcommand{\mat}[1]{\ensuremath{\mathbf{#1}}\xspace}
\newcommand{\ring}{\mathcal{R}}
\newcommand{\field}{\mathcal{K}}
\newcommand{\lattice}{\mathcal{L}}
\newcommand{\inner}[2]{\langle #1, #2 \rangle}
\newcommand{\transpose}{\mathtt{T}}
\newcommand{\tuple}[1]{\ensuremath{\left(#1\right)}}
\newcommand{\diag}[1]{\ensuremath{\mathsf{diag}\left(#1\right)}}
\newcommand{\Span}{\mathsf{span}}

\newcommand{\TrapGen}{\mathsf{TrapGen}}
\newcommand{\SampPre}{\mathsf{SampPre}}
\newcommand{\Hyb}{\mathsf{Hyb}}

\newcommand{\SIS}[1]%
{\ifthenelse{\equal{#1}{}}%
  {\ensuremath{\mathsf{SIS}}}%
  {\ensuremath{\mathsf{SIS}_{\/#1}}}%
  \xspace%
}
\newcommand{\LWE}[1]%
{\ifthenelse{\equal{#1}{}}%
  {\ensuremath{\mathsf{LWE}}}%
  {\ensuremath{\mathsf{LWE}_{\/#1}}}%
  \xspace%
}

\newcommand{\ctxt}{\mathsf{ctxt}}
\newcommand{\conductor}{\mathfrak{f}}

\newtheorem{definition}{Definition}
\newtheorem{theorem}[definition]{Theorem}

% %% PLOTS 

% \usepackage{tikz,pgfplots}
% \pgfplotsset{compat=newest}
% \usetikzlibrary{calc}
% \usetikzlibrary{arrows}

% \usetikzlibrary{external}
% \tikzexternalize[prefix=plots/]
% \tikzset{external/optimize=false}
% \tikzset{external/export=false}

% \newcommand{\tikzexternalizemaybe}{%
%   \ifthenelse{\equal{\buildexternal}{1}}%
%   {\tikzset{external/export=true}}%
%   {\tikzset{external/export=false}%
% }}%

% % from pgfplotsthemetol.sty
% \definecolor{DarkPurple}{HTML}{332288}
% \definecolor{DarkBlue}{HTML}{6699CC}
% \definecolor{LightBlue}{HTML}{88CCEE}
% \definecolor{DarkGreen}{HTML}{117733}
% \definecolor{DarkRed}{HTML}{661100}
% \definecolor{LightRed}{HTML}{CC6677}
% \definecolor{LightPink}{HTML}{AA4466}
% \definecolor{DarkPink}{HTML}{882255}
% \definecolor{LightPurple}{HTML}{AA4499}
% \definecolor{DarkBrown}{HTML}{604c38}
% \definecolor{DarkTeal}{HTML}{23373b}
% \definecolor{LightBrown}{HTML}{EB811B}
% \definecolor{LightGreen}{HTML}{14B03D}

% \pgfplotsset{width=1.0\textwidth,
%   height=0.45\textwidth,
%   cycle list={%
%     solid,\\%
%     dotted,DarkBlue,very thick\\%
%     dashed,DarkGreen,thick\\%
%     loosely dotted,very thick,DarkRed\\%
%     loosely dashed,black,very thick,DarkBrown\\%
%     loosely dashdotted,darkgray,very thick,DarkTeal\\%
%     \\%
%   },
%   legend pos=north west,
%   legend style={fill=none},
%   legend cell align={left}}


% %% LISTINGS
%
% \usepackage{listings}
% \lstdefinelanguage{Sage}[]{Python}{morekeywords={True,False,sage,cdef,cpdef,ctypedef,self},sensitive=true}
% \lstset{frame=none,
%           showtabs=False,
%           showspaces=False,
%           showstringspaces=False,
%           commentstyle=\color{gray!80!black},
%           keywordstyle={\color{gray!80!black}\textbf},
%           stringstyle ={\color{gray!80!black}},
%           basicstyle=\tt\small\relax,,
%         }
%
% \usepackage{navigator}
        
%% PDF SETUP

\usepackage{fullpage}
\pagestyle{plain}
\hypersetup{colorlinks=true,citecolor=darkgray,linkcolor=darkgray}
\allowdisplaybreaks

%% REFERENCES
\usepackage[capitalize]{cleveref}

\DeclareRobustCommand{\abbrevcrefs}{%
\crefname{lemma}{Lem.}{Lems.}%
\crefname{corollary}{Cor.}{Cors.}%
\crefname{theorem}{Thm.}{Thms.}%
}

\DeclareRobustCommand{\cshref}[1]{{\abbrevcrefs\cref{#1}}}


%% OPENING


\title{Coding Workshop 2025}

\anonymous{}{
{
  \author{Russell W. F. Lai \and Ivy K. Y. Woo}
}
  % \institute{%
	% }
}

\begin{document}

\maketitle

% \begin{abstract}
%   We do stuff.
% \end{abstract}

\section{Introduction}

We write down a box-standard lattice-based encryption scheme, its correctness bound and the underlying LWE assumption.
Then, we estimate the required parameters. 
Finally, we implement the scheme.

\section{Preliminaries}

\begin{itemize}
  \item $\ring$: Cyclotomic ring with conductor $\conductor$ and degree $\varphi = \varphi(\conductor)$. We will assume that $\conductor$ is a power of two.
  \item $\norm{\cdot}$: The coefficient $\ell_\infty$-norm
  \item $\gamma \leq \varphi$: coefficient $\ell_\infty$-norm ring expansion factor 
\end{itemize}

\begin{definition}[$\LWE{}$ assumption]
  The $\LWE{\ring,n,m,q,\chi_s,\chi_e}$ assumption states that the following distributions are computationally indistinguishable: 
  \begin{align*}
    \set{
      \begin{aligned}
      &(\mat{A}, \vec{b}) \in \ring_q^{n \times m} \times \ring_q^m  \\ 
      \midrule 
      &\quad 
      \begin{aligned}
        &\mat{A} \sample \ring_q^{n \times m} \\
        &\vec{s} \sample \chi_s^n;~
        \vec{e} \sample \chi_e^m \\
        &\vec{b}^\transpose \coloneqq \vec{s}^\transpose \mat{A} + \vec{e}^\transpose \bmod q
      \end{aligned}
      \end{aligned}
    }
    && \text{and} && 
    \set{
      \begin{aligned}
      &(\mat{A}, \vec{b}) \in \ring_q^{n \times m} \times \ring_q^m  \\ 
      \midrule 
      &\quad 
      \begin{aligned}
        &\mat{A} \sample \ring_q^{n \times m} \\
        &\vec{b} \sample \ring_q^m
      \end{aligned}
      \end{aligned}
    }
  \end{align*}
\end{definition}

\section{Regev/Lindner-Peikert Encryption}

\begin{figure}[h]
  \begin{pchstack}[center,boxed,space=1em]
    \procedure[]{$\kgen(\secparam)$}{
      \mat{A} \sample \ring_q^{n \times n} \\
      \mat{S}, \mat{E} \sample \chi^{\ell \times n} \\
      \mat{B} \coloneqq \mat{S} \mat{A} + \mat{E} \bmod q \in \ring_q^{\ell \times n} \\ 
      \pk \coloneqq (\mat{A}, \mat{B}) \\ 
      \sk \coloneqq \vec{s} \\ 
      \pcreturn (\pk, \sk)
    }
    \procedure[]{$\enc(\pk, \vec{m} \in \ring_p^\ell)$}{
      \vec{r}, \vec{f} \sample \chi^n;~
      \vec{g} \sample \chi^\ell \\
      \vec{c} \coloneqq \mat{A} \vec{r} + \vec{f} \bmod q \in \ring_q^n \\
      \vec{d} \coloneqq \mat{B} \vec{r} + \vec{g} + \floor{\frac{q}{p}} \cdot \vec{m} \bmod q \in \ring_q^\ell \\
      \ctxt \coloneqq (\vec{c}, \vec{d}) \in \ring_q^n \times \ring_q^\ell \\ 
      \pcreturn \ctxt
    }
    \procedure[]{$\dec(\sk, \ctxt)$}{
      \tilde{\vec{m}} \coloneqq \vec{d} - \mat{S} \vec{c} \bmod q \\
      \vec{m} \coloneqq \left\lfloor \frac{p}{q} \cdot \tilde{\vec{m}} \right\rceil \bmod p \\ 
      \pcreturn \vec{m}
    }
  \end{pchstack}
\end{figure}

\begin{theorem}
  Let $\chi$ be any distribution over $\ring_{2b+1}$ for some $b \in \NN$.
  If $q \geq 4 p \varphi n b^2 + 2 p b$ then the above PKE scheme is correct.
\end{theorem}

\begin{proof}
  Note that $\tilde{\vec{m}} = \vec{d} - \mat{S} \vec{c} = \floor{\frac{q}{p}} \cdot \vec{m} + \mat{E} \vec{r} + \vec{g} - \mat{S} \vec{f} \bmod q$.
  Decryption is correct as long as $\norm{\mat{E} \vec{r} + \vec{g} - \mat{S} \vec{f}} < \frac{q}{2p}$.
  Note that $\norm{\mat{S}}, \norm{\mat{E}}, \norm{\vec{f}}, \norm{\vec{g}} \leq b$, hence $\norm{\mat{E} \vec{r}}, \norm{\mat{S} \vec{f}} \leq \gamma n b^2$.
  Also, $\gamma \leq \varphi$.
  Therefore, $\norm{\mat{E} \vec{r} + \vec{g} - \mat{S} \vec{f}} \leq 2 \varphi n b^2 + b < \frac{q}{2p}$.
\end{proof}

\begin{theorem}
  If the $\LWE{\ring,n,n,q,\chi,\chi}$ and $\LWE{\ring,n,n+\ell,q,\chi,\chi}$ assumptions hold, then the above PKE scheme is IND-CPA-secure.
\end{theorem}

\begin{proof}
  By the $\LWE{\ring,n,n,q,\chi,\chi}$ assumption, $\pk = (\mat{A}, \mat{B})$ is indistinguishable from uniform.
  Then, by the $\LWE{\ring,n,n+\ell,q,\chi,\chi}$ assumption, $\ctxt = (\vec{c}, \vec{d})$ is indistinguishable from uniform.
\end{proof}

\section{Parameter Estimation}

We use the Lattice Estimator to estimate parameters of the scheme. 
Note that the Lattice Estimator ignores the ring structure of $\LWE{}$ over $\ring$, i.e. $\LWE{\ring,n,m,q,\chi,\chi}$ is treated as $\LWE{\ZZ,\varphi n,\varphi m,q,\chi,\chi}$.
Our task is to pick $\varphi, n, \ell, p, q, \chi$ such that: 
\begin{itemize}
  \item The message space is at least $2^{256}$.
  \item The correctness bound holds. 
  \item $\LWE{\ring,n,n,q,\chi,\chi}$ and $\LWE{\ring,n,n+\ell,q,\chi,\chi}$ each have at least 128-bit estimated security.  Note that 
\end{itemize}

We will fix $\varphi = 256$, $\ell = 1$ and $p = 2$, so that the message space is $|\ring_p^\ell| = 2^{256}$.
We fix $\chi$ to be the uniform distribution over $\ring_3$ (i.e. $b = 1$).  
To pick $n, q$, we will use the following strategy: 
\begin{enumerate}
  \item Pick a candidate $n$.
  \item Pick $q$ just large enough so that the correctness bound is satisfied. This means picking $q = 8 \varphi n + 5$. 
  \item Run the Lattice Estimator on $\LWE{\ring,n,n+\ell,q,\chi,\chi}$. (There is no need to run the Lattice Estimator on $\LWE{\ring,n,n,q,\chi,\chi}$ because it is no easier that $\LWE{\ring,n,n+\ell,q,\chi,\chi}$.)
  \item If $\LWE{\ring,n,n+\ell,q,\chi,\chi}$ has at least 128-bit estimated security, we are done. Otherwise, repeat Step 1 with a bigger $n$.
\end{enumerate}


\clearpage
\bibliographystyle{alpha}
\bibliography{abbrev3,crypto,local}

\appendix
\clearpage


\end{document}

%%% Local Variables:
%%% mode: latex
%%% TeX-master: t
%%% TeX-command-extra-options: "-shell-escape"
%%% eval: (add-hook 'after-save-hook (lambda nil (interactive) (save-window-excursion (when (and (fboundp 'reftex-toc-rescan) (eq major-mode 'latex-mode)) (reftex-toc-rescan)))) 0 t)
%%% End:
